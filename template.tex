%%%%%%%%%%%%%%%%%%%%%%%%%%%%%%%%%%%%%%%%%%%%%%%%%%%%%%%%%%%%%%%%%%%%%%%
%                          template.tex
%
% LaTeX template for papers conforming to the United States Sections of
% the Combustion Institute style guide.
%
% Authors:
%     Bryan W. Weber, University of Connecticut
%     Kyle E. Niemeyer, Oregon State University
%.   (Template slightly modified for ESSCI meetings by Perrine Pepiot, Cornell University)
%
% This work is licensed under the Creative Commons Attribution 4.0
% International License. To view a copy of this license, visit
% http://creativecommons.org/licenses/by/4.0/.
%
% The source for this template can be found at
% https://github.com/pr-omethe-us/ussci-latex-template
%%%%%%%%%%%%%%%%%%%%%%%%%%%%%%%%%%%%%%%%%%%%%%%%%%%%%%%%%%%%%%%%%%%%%%%
\documentclass[12pt]{essci}

%======================================================================
% Remove this in the real document
\usepackage{blindtext}
%======================================================================
% BibLaTeX and biber (not BibTeX) are used to process the references,
% so these packages must be installed on your system. All configuration
% for the bibliography and citations are done in the essci.cls file
% Add your bibliography file here, replacing template.bib
\addbibresource{template.bib}
%======================================================================
% Replace "Reaction Kinetics" in the line below by your paper topic
\newcommand\papertopic{Reaction Kinetics}
%======================================================================

\title{ Title of Paper }

\author[1]{Author Name}
\author[1]{Author Name}
\author[2]{Author Name}
\author[2,*]{Author Name}

\affil[1]{Department, Institution, Address, Country}
\affil[2]{Department, Institution, Address, Country}
\affil[*]{Corresponding author: \email{author@university.edu}}

\begin{document}
\maketitle

%====================================================================
\begin{abstract} % not to exceed 200 words
Abstract should be between 150--200 words and should state briefly the purpose
of the research, the principal results and major conclusions. An abstract is
often presented separately from the article, so it must be able to stand alone.
For this reason, references should be avoided, but if essential, then cite the
author(s) and year(s). Also, non-standard or uncommon abbreviations should be
avoided, but if essential they must be defined at their first mention in the
abstract itself.
\end{abstract}

% Provide 2-4 keywords describing your research. Only abbreviations firmly
% established in the field may be used. These keywords will be used for
% sessioning/indexing purposes. Use \sep between each keyword.
\begin{keyword}
    Keyword1\sep Keyword2\sep Keyword3\sep Keyword4
\end{keyword}

%====================================================================
\section{Introduction}

You can use the following cite commands:

Single reference with number only: \cite{Zhao2013}

Multiple references with number only: \cite{Affleck1967,Turanyi2014,cantera}

Single reference with two or fewer authors: \textcite{Affleck1967}

Single reference with three or more authors: \textcite{Wang2011}

Two references with authors: \textcite{Kee1996,Baumgardner2013}

Three or more references with authors: \textcite{Kee1996,Baumgardner2013,Haworth2011}

\section{Methods/Experimental}
%
\blindtext


\section{Results and Discussion}
%
\blindtext

\subsection{Results}
%
\blindtext

\section{Conclusions}
%
\blindtext

\blindtext

\section{Acknowledgements}
This research was funded by \ldots

\noindent\textbf{Page Limits:} The total length of the paper including references should be limited to 10 pages.

\printbibliography

\end{document}
